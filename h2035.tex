\documentclass{article}

\usepackage[utf8]{inputenc}
\usepackage[T1]{fontenc}
\usepackage[french]{babel}
\usepackage{times}
\usepackage{amsmath}
\usepackage{amssymb}
\usepackage{fancybox}            %For \ovalbox

\title{Travail pratique \#2}
\author{IFT-2035}

\newcommand \kw [1] {\textsf{#1}}
\newcommand \id [1] {\textsl{#1}}
\newcommand \punc [1] {\kw{`#1'}}
\newenvironment{outitemize}{
  \begin{itemize}
  \let \origitem \item \def \item {\origitem[]\hspace{-18pt}}
}{
  \end{itemize}
}
\newcommand \Align [2][t] {\begin{array}[#1]{@{}l@{}} #2 \end{array}}
\newcommand \MAlign \Align
\newcommand \Jop  [3] {\vdash#1\;:\;#2\;\Rightarrow\;#3}
\newcommand \Jval [2] {\vdash #1 : #2}
\newcommand \Jtype [3][\Gamma] {#1 \vdash #2 : #3}
\renewcommand \: {\!:\!}
\newcommand \uPTS {{\ensuremath{\mu PTS}}}

\begin{document}

\maketitle

%% {\centering \ovalbox{\large ¡¡ du le 3 décembre à 23h59 !!} \\}

\section{Survol}

Ce TP a pour but de vous familiariser avec le langage Prolog, ainsi qu'avec
l'inférence de types.
Comme pour le TP précédent, les étapes sont les suivantes:
\begin{enumerate}
\item Parfaire sa connaissance de Prolog.
\item Lire et comprendre cette donnée.
\item Lire, et comprendre le code fourni.
\item Compléter le code fourni.
\item Écrire un rapport.  Il doit décrire \textbf{votre} expérience pendant
  les points précédents: problèmes rencontrés, surprises, choix que vous
  avez dû faire, options que vous avez sciemment rejetées, etc...  Le
  rapport ne doit pas excéder 5 pages.
\end{enumerate}

Ce travail est à faire en groupes de 2 étudiants.  Le rapport, au format
\LaTeX\ exclusivement (compilable sur \texttt{ens.iro}) et le code sont
à remettre par remise électronique avant la date indiquée.  Aucun retard ne
sera accepté.  Indiquez clairement votre nom au début de chaque fichier.

Si un étudiant préfère travailler seul, libre à lui, mais l'évaluation de
son travail n'en tiendra pas compte.  Des groupes de 3 ou plus
sont \textbf{exclus}.

\section{Le langage H2035}

Vous allez écrire un programme Prolog qui va manipuler des expressions du
langage du futur H2035.  C'est un langage fontionnel typé statiquement, avec
surcharge et polymorphisme paramétrique similaire à Haskell.
Votre travail sera d'implanter la phase d'élaboration qui fait l'inférence
de types, l'élimination du sucre syntaxique, et la conversion vers DeBuijn,
ainsi que l'évaluateur.

La syntaxe du langage est la suivante:
\begin{displaymath}
  \begin{array}{c@{~::=~}l}
    \tau & \kw{int} ~\mid~ \kw{bool} ~\mid~ \kw{list}(\tau)
    ~\mid~ \tau_1 \to \tau_2 ~\mid~ \alpha ~\mid~ \kw{forall}(\alpha,\tau) \\
    e & \MAlign{
      c ~\mid~ x ~\mid~ \kw{lambda}(x,e) ~\mid~ \kw{app}(e_1,e_2)
      ~\mid~ x(e_1,..,e_n) ~\mid~ \\
      \kw{if}(e_1,e_2,e_3) ~\mid~ \kw{let}(\id{rdecls},e) ~\mid~(e : \tau)~\mid~ {?}
    } \smallskip \\
    \id{rdecls} & \id{rdecl} ~\mid~ \id{rdecl}, \id{rdecll}  \smallskip\\
    \id{rdecl} & \id{decl} ~\mid~ [ \id{decls} ] \smallskip\\
    \id{decls} & \id{decl} ~\mid~ \id{decl}, \id{decls}  \smallskip\\
    \id{decl} & x = e ~\mid~ x(x_1,..,x_n) = e
  \end{array}
\end{displaymath}
où \kw{forall} est le type donné aux expressions polymorphes.
$c$ représente n'importe quelle constante prédéfinie, cela inclut par
exemple les nombres entiers et les opérations sur les listes; \kw{lambda} et
\kw{app} définissent et appellent les fonctions, et \kw{let} introduit un
ensemble de déclarations locales, où celles qui sont placées entre $[...]$
peuvent être (mutuellement) récursives (d'où le $r$ de \id{rdecl}).

Comme on le voit, la syntaxe n'inclut presqu'aucune annotation de types, car
les types seront complètement inférés, tout comme en Haskell.  La seule
annotation de type est de la forme $(e : \tau)$, qui permet d'aider l'inférence
lorsque nécessaire ou de vérifier que l'inférence trouve le bon type.
En plus de cela, la syntaxe inclut aussi le terme $?$ qui fait l'inverse
de l'inférence de type, c'est à dire que le code sera inféré sur la base du
type attendu.

La syntaxe du langage repose sur la syntaxe des termes de Prolog, donc par
exemple, $x = e$ peut aussi s'écrire $\kw{=}(x,e)$, et $+(e_1,e_2)$ peut
s'écrire $e_1 + e_2$, et votre code n'a pas à s'en soucier car ces
différences n'affectent que l'analyseur syntaxique, qui est déjà fourni
par Prolog lui même.

Le langage inclut parmis ses primitives des opérations sur les listes dont
le nom dérive de ceux utilisés en Lisp: \id{cons}, \id{car}, \id{cdr},
\id{nil}.  À l'exécution, ces listes sont représentées par des
listes Prolog.

\section{Élaboration}

Après l'analyse syntaxique (qu'on ne voit pas dans le code vu qu'elle est
faite par Prolog), le code passe par une phase d'\emph{élaboration} qui va
implémenter certaines fonctionnalités du langages: l'élimination du sucre
syntaxique, l'inférence des types (et du code), et la ``désambiguation''
des identificateurs.

\subsection{Sucre syntaxique}

H2035 utilise un peu de sucre syntaxique que la phase d'élaboration va
devoir éliminer.  Plus précisément, les règles d'équivalence suivantes
devront être utilisées:
\begin{displaymath}
  \begin{array}{r@{~\Longrightarrow~}l}
    f(e_1,...,e_n) & \kw{app}(...\kw{app}(f,e_1)...,e_n) \\
    f(x_1,...,x_n) = e & f = \kw{lambda}(x_1,...\kw{lambda}(x_n,e)...) \\
    \kw{let}(\id{decl}, e) & \kw{let}([\id{decl}], e) \\
    \kw{let}(d_1,...,d_n, e) & \kw{let}(d_1,...\kw{let}(d_n,e)...)
  \end{array}
\end{displaymath}

\subsection{Surcharge}

H2035, contrairement à Psil, Slip, et Haskell, offre ce qu'on appelle la
\emph{surcharge}, en anglais \emph{overloading}.  Cette fonctionnalité
permet d'avoir plusieurs définitions actives pour un même identificateur.
Habituellement, lorsqu'un identificateur est défini plusieurs fois, la
définition la plus proche cache les autres (en anglais, on utilise le terme
\emph{shadowing}), mais en H2035 ce n'est pas le cas.  Ainsi,
chaque usage d'un identificateur introduit une ambiguïté s'il y a plusieurs
définitions qui y corresponent.  L'élaboration va résoudre ces ambiguïtés
en utilisant les informations de types.  Exemple:
%%
\begin{verbatim}
  let(x = 2,
      x(i) = i + x,
      x(l) = cons(3,l),
      x + x(car(x(nil))))
\end{verbatim}
%%
Ces trois définitions de $x$ ont chacune un type différent, et l'élaboration
va donc pouvoir utiliser cette information de typage pour savoir que dans
l'expression finale, chacun des trois $x$ fait référence à une
autre définition.

L'expression $?$ généralise cette idée au cas où le nom de la variable n'est
pas spécifié et l'élaboration peut utiliser n'importe quelle variable dont
le type concorde.  Bien sûr, on s'attend à ce que le programmeur n'utilise
ce terme que dans le cas où il n'y a qu'une seule variable du bon type.

Le code renvoyé par le prédicat \id{elaborate} (et attendu par \id{eval})
n'a plus de telles ambiguïtés, car il utilise ce qu'on appelle les
\emph{indices de de Bruijn} %% (comme vous l'avez fait au TP1)
plutôt que des variables avec des noms.  Plus précisément, il
doit utiliser la syntaxe suivante:
%%
\begin{displaymath}
  \begin{array}{c@{~::=~}l}
    e & n ~\mid~ \kw{var}(i) ~\mid~ \kw{lambda}(e) ~\mid~ \kw{app}(e_1,e_2)
    ~\mid~ \kw{if}(e_1,e_2,e_3) ~\mid~ \kw{let}([es],e) \smallskip\\
    es & e ~\mid~ e, es
  \end{array}
  %% ~\mid~ \kw{tfn}(\alpha,e) ~\mid~ \kw{tapp}(\tau,e)
\end{displaymath}
%
où $\kw{var}(i)$ est une référence à la $i$-ème variable de l'environnement
(en comptant comme 0 la variable la plus récemment ajoutée).
Ainsi le code ci-dessus deviendra:
\begin{verbatim}
  let([2],
      let([lambda(app(app(var(Np1), var(0)), var(2)))],
          let([lambda(app(app(var(Nc), 3), var(0)))],
              app(app(var(Np), var(2)), ...))))
\end{verbatim}
où \texttt{Np1}, \texttt{Nc}, et \texttt{Np2} sont les entiers correspondants
à la position (ou la profondeur, selon comment on le regarde) respective de
\id{cons}, et $+$ dans l'environnement.  Notez que \texttt{Np1} et
\texttt{Np2} font référence à $+$ mais ne seront pas égaux; plus
spécifiquement, $\texttt{Np2} = 1+\texttt{Np1}$ vu que l'environnement au
deuxième usage du $+$ contient une variable supplémentaire (le 3$^e$ $x$)
par rapport à l'environnement du premier usage du $+$.

\subsection{Inférence de types}

\newcommand \CheckKind [1] {\vdash #1}
\newcommand \CheckType [2][\Gamma] {#1 \vdash #2 ~:~}
\newcommand \InferType [2][\Gamma] {#1 \vdash #2 ~:~}
\newcommand \BothType [2][\Gamma] {#1 \vdash #2 ~:~}
\newcommand \Axiom [2] { \mbox{}\hspace{10pt}\nolinebreak\ensuremath{\displaystyle
    \frac{#1}{#2}}\nolinebreak\hspace{10pt}\mbox{} \medskip}
\newcommand \HA {\hspace{13pt}}
\renewcommand \: {\!:\!}

\begin{figure}
  \begin{minipage}{\columnwidth}
    \noindent \centering

    \Axiom{}{\InferType{n}{\kw{int}}}
    \Axiom{x\:\tau \in \Gamma}{\InferType{x}{\tau}}
    \Axiom{\InferType{e}{\kw{bool}}
           \HA \CheckType{e_t}{\tau} \HA
           \CheckType{e_e}{\tau}}
          {\CheckType{\kw{if}(e,e_t,e_e)} {\tau}} \bigskip
    \Axiom{\CheckType[\Gamma,x\:\tau_1]{e}{\tau_2}}
          {\CheckType{\kw{lambda}(x,e)}{\tau_1\to\tau_2}}
    \Axiom{\InferType{e_1}{\tau_1\to\tau_2}
           \HA
           \CheckType{e_2}{\tau_1}}
          {\InferType{\kw{app}(e_1,e_2)}{\tau_2}}
    \Axiom{\CheckType{e}{\tau}}
          {\InferType{(e : \tau)}{\tau}}
    \Axiom{x\:\tau \in \Gamma}
          {\CheckType{{?}}{\tau}}
    \Axiom{
        \InferType[\Gamma,x_1\:\tau_1]{e_1}{\tau_1} \HA
        \InferType[\Gamma,x_1\:\tau_1]{e}{\tau}
    } {
      \InferType{\kw{let}([x_1=e_1],e)}{\tau}
    }
    \mbox{}
  \end{minipage}
  \caption{Règles de typage simple}
  \label{fig:typing}
\end{figure}

H2035 utilise un typage statique similaire à celui de Haskell, où les types
sont inférés.  Dans un premier temps, je vous recommande de faire une
inférence simple, sans polymorphisme (ni récursion mutuelle), puis dans un
deuxième temps vous ajouterez ce qu'on appelle le \emph{let-polymorphisme},
en utilisant l'algorithme d'inférence de types de Hindley-Milner.

Les règles de typage sans polymorphisme sont données en
Fig.~\ref{fig:typing}.  Dans ces règles, le jugement $\InferType e \tau$
signifie que l'expression $e$ a type $\tau$ dans un contexte $\Gamma$ qui
donne le type de chaque variable dont la portée couvre $e$.

Comme vous pouvez le voir, la règle du \kw{let} n'accepte pas que le
\kw{let} déclare simultanément plus qu'un identificateur, et
n'autorise donc pas la récursion mutuelle.  Vous pouvez aussi remarquer que
ces règles présument que le sucre syntaxique a déjà été éliminé, c'est
pourquoi il n'y a pas de règle par exemple pour la forme $x(e_1,..,e_n)$.

\subsubsection{Récursion mutuelle et inférence du polymorphisme}

\begin{figure}
  \begin{minipage}{\columnwidth}
    \noindent \centering
    \Axiom{}{\tau \subset \tau}
    \Axiom{\sigma[\tau'/\alpha] \subset \tau}
          {\kw{forall}(\alpha,\sigma) \subset \tau}
    \Axiom{x\:\sigma \in \Gamma
           \HA  \sigma \subset \tau}
          {\InferType{x}{\tau_2}}

    \Axiom{
      \begin{array}{c}
        \forall i \HA
        \InferType[\Gamma,x_1\:\tau_1,...,x_n\:\tau_n]{e_i}{\tau_i}
        \HA \sigma_i = \kw{gen}(\Gamma,\tau_i) \\
        \InferType[\Gamma,x_1\:\sigma_1,...,x_n\:\sigma_n]{e}{\tau}
      \end{array}
    } {
      \InferType{\kw{let}([x_1=e_1,...,x_n=e_n],e)}{\tau}
    }

    \Axiom{\{t_1,...,t_n\} = \kw{fv}(\tau) - \kw{fv}(\Gamma)}
          {\kw{gen}(\Gamma,\tau) = \kw{forall}(t_1,...\kw{forall}(t_n,\tau)..)}
    \mbox{}
  \end{minipage}
  \caption{Règles additionnelles de typage polymorphe}
  \label{fig:typing-poly}
\end{figure}

La deuxième étape introduit le let-polymorphisme de Hindley-Milner qui va
donc automatiquement inférer quand une définition est polymorphe, ainsi que
spécialiser automatiquement une définition polymorphe chaque fois qu'elle
est utilisée.  Cette étape ajoute aussi les définitions récursives.

Les nouvelles règles de typage de la deuxième étape sont données en
Fig.~\ref{fig:typing-poly}.  Ces règles remplacent la règle de typage des
variables, et la règle de typage du \kw{let}.  Dans ces nouvelles règles,
$\sigma$ est utilisé pour indiquer un type qui peut être \emph{polymorphe}
(i.e. avec un \kw{forall}), alors que $\tau$ représente un type
\emph{monomorphe} (i.e. sans \kw{forall}).

Dans la règle de typage des variable, le $\sigma \subset \tau$ indique que
$\tau$ est une instance de $\sigma$, par exemple, $\sigma$ pourrait être
$\kw{forall}(\alpha,\alpha\to\alpha))$, et $\tau$ pourrait être
$\kw{int}\to\kw{int}$ qui est l'instance correspondant au choix de \kw{int}
pour $\alpha$.  C'est donc dans cette règle que les \kw{forall} sont
\emph{éliminés}.

La règle de typage du \kw{let} utilise une fonction auxiliaire \kw{gen} qui
va \emph{généraliser} une définition, en découvrant son polymorphisme.
Cela fonctionne simplement en regardant dans le type $\tau$ de la définition
quelles sont les variables non-instanciées (ou libres: \kw{fv} veut dire
``free variables''): celles-ci peuvent être généralisées (sauf si elles
apparaissent aussi dans le contexte $\Gamma$), donnant alors un
type polymorphe.  C'est donc cette règle qui \emph{introduit}
le \kw{forall}.

Cette nouvelle règle du \kw{let} autorise non seulement l'introduction de
plusieurs définitions simultanées ($x_1...x_n$), mais autorise aussi la
récursion mutuelle.  Cela se voit dans le fait que $e_i$ est typé dans un
environnement qui n'est pas limité à $\Gamma$ mais inclus aussi les
identificateurs $x_1...x_n$, ce qui autorise donc $e_i$ à faire des
références récursives non seulement à $x_i$ mais aussi à n'importe lequel
des autres identificateurs définis par ce \kw{let}.

Détail important pour la règle du \kw{let}: si on a par exemple les
déclarations $y = x, x = 1$ et on infère le type de $y$ avant d'inférer
le type de $x$, le type de $y$ sera incomplet tant que le type de $x$ ne
sera pas inféré.  Pour cette raison, il est important d'inférer le type de
tous les $e_i$ \emph{avant} de les passer à \kw{gen}, faute de quoi,
\kw{gen} risque de penser que $y$ est polymorphe.

\section{Votre travail}

Votre travail consiste à compléter le code des prédicats \id{elaborate} et
\id{eval}.  Le code qui vous est fourni contient déjà un squelette de ces
prédicats ainsi que quelques autres prédicats qui vous seront utiles
principalement pour la deuxième phase:
\begin{itemize}
\item \id{instantiate}: implémente la règle $\sigma \subset tau$.
\item \id{freelvars}: implémente la fonction \kw{fv}.
\item \id{generalize}: fait le travail de \kw{gen}.
\end{itemize}

\section{Remise}

Vous devez remettre deux fichiers, \texttt{h2035.pl} et
\texttt{rapport.tex}.  Ces fichiers seront remis sous forme électronique,
par l'intermédiare de Moodle/StudiUM.

\section{Détails}

La note est basée d'une part sur des tests automatiques, d'autre part sur la
lecture du code, ainsi que sur le rapport.  Le critère le plus important, et
que votre code doit se comporter de manière correcte.  En règle générale,
une solution simple est plus souvent correcte qu'une solution complexe.
Ensuite, vient la qualité du code: plus c'est simple, mieux c'est.
S'il y a beaucoup de commentaires, c'est généralement un symptôme que le
code n'est pas clair; mais bien sûr, sans commentaires le code (même simple)
et souvent incompréhensible.  L'efficacité de votre code est sans
importance, sauf s'il utilise un algorithme vraiment particulièrement
ridiculement inefficace.

Les points seront répartis comme suit: 25\% sur le rapport, 25\% sur les
tests automatiques, 25\% sur le code de l'élaboration, et 25\% sur le
code de l'évaluateur.

\begin{itemize}
\item Le code ne doit en aucun cas dépasser 80 colonnes.
\item Vérifiez la page web du cours, pour d'éventuels errata, et d'autres
  indications supplémentaires.
\item Tout usage de matériel (code ou texte) emprunté à quelqu'un d'autre
  (trouvé sur le web, ...) doit être dûment mentionné, sans quoi cela sera
  considéré comme du plagiat.
\end{itemize}


\end{document}
